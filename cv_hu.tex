\documentclass[a4paper,10pt]{article}
%\documentclass[a4paper,10pt]{scrartcl}

\usepackage[utf8x]{inputenc}
% \usepackage{t1enc}
\usepackage[magyar]{babel}

\title{\textbf{Önéletrajz}}
\author{Szabados Péter}
% \date{}

% \pdfinfo{%
%   /Title    (Önéletrajz)
%   /Author   (Szabados Péter)
%    /Creator  (Szabados Péter)
%    /Producer (Szabados Péter)
%    /Subject  (Önéletrajz)
%    /Keywords ()
% }

\newcommand{\pont}[1]{\emph{#1}}
\addtolength{\textheight}{2cm}
\sloppy

\usepackage[pdftex,colorlinks]{hyperref}

\begin{document}
\maketitle

\section*{Személyes adatok}
\begin{tabular}{ll}
 \pont{Születés}&Budapest, 1985. augusztus 29.\\
 \pont{Értesítési cím}&1022 Budapest Felvinci út 11.\\
 \pont{E-mail}&kangirigungi@gmail.com\\
 \pont{Telefon}&30/708-7865\\
 \pont{LinkedIn profil}&\href{https://www.linkedin.com/profile/view?id=211928596}{https://www.linkedin.com/profile/view?id=211928596}\\
 \pont{GitHub profil}&\href{https://github.com/petersohn}{https://github.com/petersohn}\\
 \pont{StackOverflow profil}&\href{http://stackoverflow.com/users/294813/petersohn}{http://stackoverflow.com/users/294813/petersohn}\\
\end{tabular}

\section*{Tanulmányok}
\begin{tabular}{ll}
%  2009 október--(2010 október)&2F Iskola Webprogramozó tanfolyam\\
 2004--2009
  &\textbf{Budapesti Műszaki és Gazdaságtudományi Egyetem}\\
  &Villamosmérnöki és Informatikai Kar\\
  &Műszaki Informatika szak\\
  &Integrált Intelligens Rendszerek szakirány\\
  &Diplomamunka témája:\\
  &\emph{Valószínűségi tudásfúzió genetikai asszociációs kutatásokban}\\
  1998--2004&\textbf{Móricz Zsigmond Gimnázium, Budapest}
\end{tabular}

\section*{Szakmai tapasztalat}
\begin{tabular}{lp{10cm}}
 2017&Graphisoft programozó verseny: 3. helyezés\\
 2017&Craft konferencia\\
 2016&Craft konferencia, egy előadó mentorálása\\
 2015&NNG Grand Prix programozó verseny: 4. helyezés\\
 2014&NNG Grand Prix programozó verseny: 4. helyezés\\
 2010--&Ericsson Magyarország kft.\\
  &\emph{Főbb feladatok:} Nagy rendelkezésre állású, elosztott adatbázis fejlesztése, karbantartása.\\
  &\emph{Egyéb feladatok:} Ericsson Programozó Bajnokság 2011 és 2012: versenyfeladatok tervezése és ellenőrzése.\\
 2008--2009&Rubin Zrt.\\
  &\emph{Főbb feladatok:} Delphi komponensek fejlesztése.
\end{tabular}

\section*{Számítástechnikai ismeretek}
\begin{tabular}{lp{12cm}}
 \pont{Programozás}&C, C++ (C++11, C++14, Boost), Python, Java (Java SE, Android), Delphi, Bash, Perl, SQL, HTML, CSS, JavaScript, PHP, Prolog.\\
 &OOP, TDD, Agile.\\
 \pont{Fejlesztőeszközök}&Git, Clearcase, Make, Tup, Eclipse, Vim, Jenkins\\
 \pont{Grafika}&Adobe Photoshop, GIMP, Inkscape\\
 \pont{Egyéb}&Linux, MS windows, Word, Excel, PowerPoint, OpenOffice.org, LaTeX
\end{tabular}

\section*{Nyelvtudás}
\begin{tabular}{ll}
 \pont{Angol}&Középfok, C típusú nyelvvizsga (2002). Rendszeres használat, gyakorlat munkahelyen.\\
 \pont{Német}&Alapfok (nincs nyelvvizsga)
\end{tabular}

\section*{Személyiségjegyek}
Szeretek gondolkodást igénylő feladatokat megoldani. Szeretek új, modern technológiákkal megismerkedni. Szoftverfejlesztésben elsőnek tartom a minőséget.



\end{document}
